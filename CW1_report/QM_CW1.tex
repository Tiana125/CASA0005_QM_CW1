\documentclass[12pt]{article}
\usepackage[utf8]{inputenc}
\usepackage{amsmath,amssymb}
\usepackage{amsthm}
\usepackage[nice]{nicefrac}
\usepackage{booktabs}
\usepackage{graphicx}
\usepackage{caption}
\usepackage{lmodern,textcomp}
\usepackage{float}
\usepackage{url}
\usepackage{tabularx}
\renewcommand{\tabularxcolumn}[1]{>{\small}m{#1}}
\usepackage{romannum}
\bibliographystyle{plain}
\usepackage{parskip}
\setlength{\parindent}{0em}
\linespread{1.25}
\newcommand\diff{\,\mathrm{d}}
\renewcommand{\thesubsection}{\arabic{subsection}.}
\renewcommand{\thesubsubsection}{\roman{subsubsection}.}

\usepackage[margin=1.2in]{geometry}
\usepackage{siunitx}
\usepackage[numbers,sort&compress]{natbib}
\usepackage{xcolor}
\usepackage{listings}
\lstdefinestyle{lfonts}{
  basicstyle   = \footnotesize\ttfamily,
  stringstyle  = \color{purple},
  keywordstyle = \color{blue!60!black}\bfseries,
  commentstyle = \color{olive}\scshape,
}
\lstdefinestyle{lnumbers}{
  numbers     = left,
  numberstyle = \tiny,
  numbersep   = 1em,
  firstnumber = 1,
  stepnumber  = 1,
}
\lstdefinestyle{llayout}{
  breaklines       = true,
  tabsize          = 2,
  columns          = flexible,
}
\lstdefinestyle{lgeometry}{
  xleftmargin      = 20pt,
  xrightmargin     = 0pt,
  frame            = tb,
  framesep         = \fboxsep,
  framexleftmargin = 20pt,
}
\lstdefinestyle{lgeneral}{
  style = lfonts,
  style = lnumbers,
  style = llayout,
  style = lgeometry,
}
\lstdefinestyle{python}{
    language = {Python},
    style    = lgeneral,
}

\begin{document}
\pagenumbering{arabic}
\title{How does allocation of funding to London local authorities affect obesity cases in 2018}
\author{CASA 0005 Quantitative Methods Coursework 1}
\date{November 15, 2020}
\maketitle
\subsection{Introduction}
Today, almost one in five Year 6 children in the UK is found to be obese, and sadly this number is not dropping during the past few years \citep{nhs_national_2018}. The origins of childhood obesity stem from various aspects, including lifestyle issues, genetic and environmental causes. The government has been taking considerable forms of actions to tackle this problem from its root, from sugar reduction to advertising and promotions. This study investigates how government has been allocating their funding to local authorities in London.
% , and the performance of local authorities using said funding. 


\subsection{Data}
The data used in this study contains population, obesity cases, total budget, and allocation of funding for local authorities across London in 2018. An illustration of data employed is shown in Table \eqref{t1}. 
\begin{table}[H]

\begin{center}
\resizebox{\textwidth}{!}{%
\begin{tabular}{c|c|c|c|c|c|c}%{36em}
% {|*{15}{>{\centering\arraybackslash}X|}}
\hline
% \multicolumn{1}{c|}{} & \multicolumn{12}{c|}{data} & \multicolumn{1}{c|}{mean} & \multicolumn{1}{c}{uncertainties} \\ \hline
 & local authorities & total obesities & total population & ... & total budget & ... \\ \hline
0 & Barking and Dagenham & 763 & 181779 & ... & 139000 & ... \\ \hline
1 & Barnet & 773 & 355955 & ... & 220000 & ... \\ \hline
 \multicolumn{7}{c}{\vdots} \\ \hline
\end{tabular}}
\captionsetup{font=scriptsize}
\caption{Illustration of data used, list of column names{} include: \newline
Names of local authority areas; total obesity cases in each area; total population; obesity density (obesity cases divided by population); total budget allocated (in pounds); percent of budget spent on improving air quality, cleaner environment, health training, raising school awareness, media awareness and subsiding counselling.
} \label{t1}
\end{center}
\end{table}

Only one observation is considered an outlier due to its relatively small scale, and therefore dropped from the dataset, City of London data. Its population is below the average population in London boroughs by $97\%$, having this data in the linear regression plot of obesity cases in 2018 vs. total budget spent lowers the regression coefficient from $0.437$ to $0.349$. 


\subsection{Methodlogy}
Three approaches were taken to investigate the criteria of funding:

\romannum{1}. 
A linear line was fitted using scipy.stats.linregress() between total budget spent and obesity density. 


\romannum{2}.
A more outlier-robust linear approach, called Random Sample Consensus (RANSAC) was used. This method compliments the ordinary least squares methods by adding detections of outliers and accord them to have no influence on the parameters of the model (Fischler and Bolles, 1981). In \verb|sklearn.linear_model.RANSACRegressor()|, outliers are classified as those whose residual exceed the median absolute deviation of dependent variables (Pedregosa, F. et al., 2011).

\romannum{3}.
Finally, a non-linear approach was taken. Polynomial regression of various degrees was fitted to the data using {\verb|numpy.polyfit()| in Python. 


\subsection{Results}
\romannum{1}. Logically one might assume that the more obesity cases discovered per unit population, the more funding would be allocated. The linear line fitted to total budget spent vs. obesity density yields a coefficient of determination of 0.002, which means that only $0.2\%$ of the variance of dependent variable (total budget) is explained by independent variable (obesity density). Log transformations were also applied to the variables, associated parameters are shown in Table \eqref{t2}. 


\begin{table}[H]
\begin{center}
\resizebox{\textwidth}{!}{%
\begin{tabular}{|l|c|c|c|c|}
\hline
 &	unchanged data &$log(x)$ and y&$log(y)$ and x&	$log(x)$ and $log(y)$	 \\ \hline
slope &	\num{3e-9} & 0.001 & \num{3e-7} & 0.0574\\ \hline
constant	&-0.002 & -0.005 & 0.47 & -0.259 \\ \hline
$R^2$	&0.037 & 0.032 &0.29 & 0.29\\ \hline
Pearson correlation coefficient &  0.192  &0.18 & 0.538 & 0.537  \\ \hline
p value & 0.29 & 0.32 & 0.001 & 0.002 \\ \hline
relationship implied & linear   & exponential & exponential  & power \\ \hline
\end{tabular}}
\captionsetup{font=scriptsize}
\caption{coefficients related to the relationship between original and log-transformed total budget spent and obesity density} \label{t2}
\end{center}
\end{table}

As the maximum value of $R^2$ being under $0.3$ in Table \eqref{t2}, the amount of funding allocated to local authorities is extremely weakly correlated with the obesity cases per unit population by the power law, thus indicating that this relationship is meaningless. 


\romannum{2}.



\begin{figure}[H]
% \begin{center}
% \includegraphics[scale=0.45]{3_i.png}
% \end{center}
\caption{Best fit line for $I$ and $V$}
\end{figure}


In Table \eqref{iv}, the expected $I$ is calculated from the equation of the fitted line, which was yielded in python programme code ``3.ii'': $I=6.777V+1.998$. 




\romannum{3}. 
\begin{figure}[H]

\caption{Best fit line with uncertainties}
\end{figure}




The probability $P(22.31;9) = 0.007953893$. As this probability is extremely close to 0, we reject null hypothesis since this implies the measurements are insignificant and unphysical. For a good fit we expect two-thirds of the data points to be within one standard error bar of the trend line. But in this graph we can only find 5 out of 11 points within one standard error bar. Therefore this is not a good fit.

This suggests that the hypothesis of linear relationship between $v$ and $i$ is rejected. This may be the case theoretically, but in the real scenario, linear relationship cannot be maintained since the resistance is not constant (could be affected by temperature).



\subsection{Best Fit by Minimizing $\chi^2$}
\romannum{1}. Brewster angle is given by $\theta_B=\arctan(\frac{n_2}{n_1})$, 
where $n_1$ is the refractive index of the initial medium through which the light propagates (the ``incident medium''), and $n_2$ is the index of the other medium. Therefore, with the knowledge of Brewster angle being 57 degree, and $n_1$ being 1 for air, we can calculate $n_2$, which is $1.539865$, rounded to $1.540$. 

The expected value of $\mathbf{R}^2$ can be calculated by 
\begin{equation}
\mathbf{R}^2_{\perp}= \left(\frac{\cos \theta_1 - n\cos \theta_2}{\cos \theta_1+n\cos \theta_2}\right)^2 \qquad \mathbf{R}^2_{\parallel}= \left(\frac{n\cos \theta_1 - \cos \theta_2}{n\cos \theta_1+\cos \theta_2}\right)^2
\end{equation}

To conclude, a table with all values needed to calculate $\chi^2$ and $P(\chi^2)$ is written below. 


\begin{table}[H]
\begin{center}
\resizebox{\textwidth}{!}{%
\begin{tabular}{|c|c|c|c|c|c|c|c|c|}
\hline
$\theta_1$	&$\theta_1$ in radian &	$\mathbf{R}^2_{\parallel}$&	$\mathbf{R}^2_{\perp}$	&$\theta_2$ in radian	&expected $\mathbf{R}^2_{\perp}$ &	expected $\mathbf{R}^2_{\parallel}$ &	$\chi^2_{\perp}$&	$\chi^2_{\parallel}$ \\ \hline
10	&0.174532925	&0.03&	0.037&	0.113008831&	0.047004879&	0.043388466&	25.02440082&	44.81275438 \\ \hline
15	&0.261799388	&0.028&	0.039&	0.168880675&	0.049393107	&0.041138201&	27.00417064	&43.1530824 \\ \hline
20	&0.34906585	&0.027&	0.043&	0.223978484	&0.052948477	&0.037974149&	24.74304991&	30.10798748 \\ \hline
25	&0.436332313	&0.025&	0.047&	0.278019262	&0.057909564&	0.033897733&	29.75464401&	19.79241518 \\ \hline
30	&0.523598776	&0.021&	0.055&	0.330698539	&0.064625378	&0.028937627&	23.16197549&	15.75148119 \\ \hline
35	&0.610865238	&0.018&	0.063&	0.381685163	&0.073594848	&0.023180871&	28.06269852&	6.710354821 \\ \hline
40	&0.698131701	&0.012&	0.075&	0.430616559	&0.085524952	&0.01682733	&27.69365481&	5.825778231 \\ \hline
45	&0.785398163	&0.007&	0.09&	0.477095003	&0.101415223	&0.010285047&	32.57682719&	2.69788409 \\ \hline
50	&0.872664626	&0.003&	0.114&	0.520685692	&0.1226792	&0.004338043&	18.83212679&	0.447589908 \\ \hline
55	&0.959931089	&0.001&	0.146&	0.560917768	&0.151316707	&0.000444236&	7.066843977&	0.077218314 \\ \hline
60	&1.047197551	&0.003&	0.188&	0.597289673	&0.190153592	&0.001271133&	1.159489483&	0.747245501 \\ \hline
65	&1.134464014	&0.012&	0.245&	0.629280412	&0.243166177	&0.011675807&	0.840726951&	0.026275222 \\ \hline
70	&1.221730476	&0.039&	0.326&	0.65636795	&0.315902862	&0.04053963	&25.48805113	&0.592615396 \\ \hline
75	&1.308996939	&0.104&	0.445&	0.67805514	&0.416000321	&0.104310777&	210.2453493&	0.024145608 \\ \hline
80	&1.396263402	&0.219&	0.604&	0.693901872	&0.553761346	&0.234129461&	630.9805948&	57.22514753 \\ \hline
85	&1.483529864	&0.467&	0.822&	0.703559953	&0.742712787	&0.490951947&	1571.615549&	143.4239389 \\ \hline
\end{tabular}}
\caption{Sets of data in order to calculate $\chi^2$} 
\end{center}
\end{table}
We omit the values for 50, 55 and 60 degrees, since these are less than 3 error bars $(3\times0.002=0.006)$ from zero.

The final result is listed in the following table. 

\begin{table}[H]
\begin{center}
\begin{tabular}{|l|c|c|}
\hline
   &     perpendicular & parallel \\ \hline
   $\chi^2$ & 2657.19 &  370.14 \\ \hline
   $P(\chi^2)$ & 0 & $4.451\times10^{-71}$  \\ \hline
\end{tabular}
\caption{list of results for 4.i} 
\end{center}
\end{table}


\romannum{2}.
Excel solver is used to minimise $\chi^2$, results are listed below.


\begin{table}[H]
\begin{center}
\begin{tabular}{|l|c|c|}
\hline
 &     perpendicular & parallel \\ \hline
   $\chi^2$ & 4793.01 &  309.32 \\ \hline
   $P(\chi^2)$ & 0 & $ 2.70\times10^{-58}$  \\ \hline
\end{tabular}
\caption{Minimising $\chi^2_{\parallel}$ (n=1.481)} 
\end{center}
\end{table}





\begin{table}[H]
\begin{center}
\begin{tabular}{|l|c|c|}
\hline
 &     perpendicular & parallel \\ \hline
   $\chi^2$ & 2251.85 &  503.80 \\ \hline
   $P(\chi^2)$ & 0 & $ 2.28\times10^{-99}$  \\ \hline
\end{tabular}
\caption{Minimising $\chi^2_{\perp}$ (n=1.582)} 
\end{center}
\end{table}

\begin{table}[H]
\begin{center}
\begin{tabular}{|l|c|c|}
\hline
 &     perpendicular & parallel \\ \hline
   $\chi^2$ & 2267.91 &  469.65 \\ \hline
   $P(\chi^2)$ & 0 & $ 4.05\times10^{-92}$  \\ \hline
\end{tabular}
\caption{Minimising both $\chi^2$ (n=1.573)} 
\end{center}
\end{table}




\appendix



\bibliography{CW1}
\end{document}